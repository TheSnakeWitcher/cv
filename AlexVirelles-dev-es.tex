\documentclass[a4paper,11pt]{article}
\usepackage{latexsym}
\usepackage{xcolor}
\usepackage{float}
\usepackage{ragged2e}
\usepackage[empty]{fullpage}
\usepackage{wrapfig}
\usepackage{lipsum}
\usepackage{tabularx}
\usepackage{titlesec}
\usepackage{geometry}
\usepackage{marvosym}
\usepackage{verbatim}
\usepackage{enumitem}
\usepackage{fancyhdr}
\usepackage{multicol}
\usepackage{graphicx}
\usepackage{cfr-lm}
\usepackage[T1]{fontenc}
\usepackage{fontawesome5}

\usepackage[hidelinks]{hyperref}
\hypersetup{
    colorlinks=true,
    linkcolor=darkblue,
    filecolor=darkblue,
    urlcolor=darkblue,
}

\usepackage[most]{tcolorbox}
\tcbset{
    frame code={},
    center title,
    left=0pt,
    right=0pt,
    top=0pt,
    bottom=0pt,
    colback=gray!20,
    colframe=white,
    width=\dimexpr\textwidth\relax,
    enlarge left by=-2mm,
    boxsep=4pt,
    arc=0pt,outer arc=0pt,
}


\definecolor{darkblue}{RGB}{0,0,139}
\setlength{\multicolsep}{0pt} 
\pagestyle{fancy}
\fancyhf{}
\fancyfoot{}
\renewcommand{\headrulewidth}{0pt}
\renewcommand{\footrulewidth}{0pt}
\geometry{left=1.4cm, top=0.8cm, right=1.2cm, bottom=1cm}
\setlength{\footskip}{5pt}
\urlstyle{same}
\raggedright
\setlength{\tabcolsep}{0in}
\titleformat{\section}{
  \vspace{-4pt}\scshape\raggedright\large
}{}{0em}{}[\color{black}\titlerule \vspace{-7pt}]

\newcommand{\resumeItem}[2]{
    \item { \textbf{#1}{\hspace{0.5mm}#2 \vspace{-0.5mm}} }
}

\newcommand{\resumePOR}[3]{
\vspace{0.5mm}\item
    \begin{tabular*}{0.97\textwidth}[t]{l@{\extracolsep{\fill}}r}
        \textbf{#1}\hspace{0.3mm}#2 & \textit{\small{#3}} 
    \end{tabular*}
    \vspace{-2mm}
}

\newcommand{\resumeSubheading}[4]{
\vspace{0.5mm}\item
    \begin{tabular*}{0.98\textwidth}[t]{l@{\extracolsep{\fill}}r}
        \textbf{#1} & \textit{\footnotesize{#4}} \\
        \textit{\footnotesize{#3}} &  \footnotesize{#2}\\
    \end{tabular*}
    \vspace{-2.4mm}
}

\newcommand{\resumeProject}[4]{
\vspace{0.5mm}\item
    \begin{tabular*}{0.98\textwidth}[t]{l@{\extracolsep{\fill}}r}
        \textbf{#1} & \textit{\footnotesize{#3}} \\
        \footnotesize{\textit{#2}} & \footnotesize{#4}
    \end{tabular*}
    \vspace{-2.4mm}
}

\newcommand{\resumeSubItem}[2]{\resumeItem{#1}{#2}\vspace{-4pt}}

\renewcommand{\labelitemi}{$\vcenter{\hbox{\tiny$\bullet$}}$}
\renewcommand{\labelitemii}{$\vcenter{\hbox{\tiny$\circ$}}$}

\newcommand{\resumeSubHeadingListStart}{\begin{itemize}[leftmargin=*,labelsep=1mm]}
\newcommand{\resumeHeadingSkillStart}{\begin{itemize}[leftmargin=*,itemsep=1.7mm, rightmargin=2ex]}
\newcommand{\resumeItemListStart}{\begin{itemize}[leftmargin=*,labelsep=1mm,itemsep=0.5mm]}

\newcommand{\resumeSubHeadingListEnd}{\end{itemize}\vspace{2mm}}
\newcommand{\resumeHeadingSkillEnd}{\end{itemize}\vspace{-2mm}}
\newcommand{\resumeItemListEnd}{\end{itemize}\vspace{-2mm}}
    \newcommand{\cvsection}[1]{
    \vspace{2mm}
    \begin{tcolorbox}
        \textbf{\large #1}
    \end{tcolorbox}
    \vspace{-4mm}
}

\newcolumntype{L}{>{\raggedright\arraybackslash}X}
\newcolumntype{R}{>{\raggedleft\arraybackslash}X}
\newcolumntype{C}{>{\centering\arraybackslash}X}

\newcommand{\socialicon}[1]{\raisebox{-0.05em}{\resizebox{!}{1em}{#1}}}
\newcommand{\ieeeicon}[1]{\raisebox{-0.3em}{\resizebox{!}{1.3em}{#1}}}

\newcommand{\headerfonti}{\fontfamily{phv}\selectfont}
\newcommand{\headerfontii}{\fontfamily{ptm}\selectfont}
\newcommand{\headerfontiii}{\fontfamily{ppl}\selectfont}
\newcommand{\headerfontiv}{\fontfamily{pbk}\selectfont}
\newcommand{\headerfontv}{\fontfamily{pag}\selectfont}
\newcommand{\headerfontvi}{\fontfamily{cmss}\selectfont}
\newcommand{\headerfontvii}{\fontfamily{qhv}\selectfont}
\newcommand{\headerfontviii}{\fontfamily{qpl}\selectfont}
\newcommand{\headerfontix}{\fontfamily{qtm}\selectfont}
\newcommand{\headerfontx}{\fontfamily{bch}\selectfont}


\begin{document}
\headerfontiii

\begin{center} {\Huge\textbf{Alejandro Virelles}} \end{center}
\vspace{-6mm}

\begin{center}
    \small{
        \socialicon{ \href{mailto:thesnakewitcher@gmail.com}{\color{black}{\faEnvelope}} }
        \socialicon{ \href{https://github.com/TheSnakeWitcher}{\color{black}{\faGithub}} }
        \socialicon{ \href{https://t.me/TheSnakeWitcher}{\color{black}{\faTelegram}} } {}
    }
\end{center}
\vspace{-6mm}

\vspace{-0.4mm}
\section{\textbf{Skills}}
\vspace{-0.4mm}
\resumeHeadingSkillStart
    \resumeSubItem{Lenguajes de programación:}
        { solidity, rust, go, typescript, python, sql, bash, lua }
    \resumeSubItem{Bases de Datos}
        { postgres, sqlite, mongodb, redis }
    \resumeSubItem{Herramientas \& Tecnologias:}
        { linux, git, github, docker, aws, grpc }
    \resumeSubItem{Lenguajes:}
        { español, inglés }
\resumeHeadingSkillEnd

\section{\textbf{Experiencia}}
\vspace{-0.4mm}
\resumeSubHeadingListStart
\resumeSubheading
    { {\href{https://portal.fusy.app}{\color{black}{Fusyona}} }}{}
    {Mid Senior Blockchain developer}{Jun 2024 - April 2025}

    \resumeItemListStart
        \item Desarrollo de contractos inteligentes con solidity siguiendo las metodologias TDD y scrum y usando
          hardhat como framework con mocha, chai, hardhat-chai-matchers, etc 

        \item Entrenando juniors para que rapidamente se vuelvan productivos, conscientes de la seguridad, sigan mejores practicas
          y desarrollen confianza para contribuir al proyecto

        \item Construyendo una \href{https://github.com/Fusyona/web3-core}{libreria} de typescript para hacer facil la creacion SDKs que se proveen al front-end para interactuar
          con contractos inteligentes

        \item Configurando un flujo de trabajo CI con github actions que detecta vulnerabilidades al crear PRs
          usando slither y corre la test suite para verificar la integridad de los cambios antes de hacer merge

        \item Construyendo factorias de tokens ERC20 para una platforma launchpad,
          y tambien factorias de ERC721(son soporte para royalties o ERC2981) y ERC1155 para
          interactuar con un marketplace on-chain 
    \resumeItemListEnd 
\resumeSubHeadingListEnd
\vspace{-6mm}


\section{\textbf{Projectos}}
\vspace{0.4mm}
\resumeItemListStart
    \vspace{0.8mm}
    \item Una \href{https://vesting.cooltech.quest}{Plataforma de vesting} para tokens erc20 con distribucion automatica que integra
      blockchain y la creacion dinamica de recursos en aws 

    \item Construyendo un \href{https://github.com/TheSnakeWitcher/AuctionHub-server}{servidor simple} en golang el cual usa el framework web echo para
      realizar operaciones CRUD usando REST API para interactuar con un servicio de subastas 

    \item Un \href{https://github.com/TheSnakeWitcher/arbitrageur-bot}{buscador} escrito en rust usando la libraria ethers-rs que observa la
      diferencia en precio de un par de tokens ERC20 y analiza estos datos para encontrar oportunidades de
      trades entre uniswap-v3 y quickswap-v3

    \item Creando un \href{https://github.com/TheSnakeWitcher/mypeople-go}{addressbook} con CLI(usando cobra), viper(para configuracion) y sqlite como base de datos(usando sqlc)

    \item Simple \href{https://github.com/TheSnakeWitcher/arbitrageur-contract}{contracto de arbitraje} escrito en solidity usando la libreria Openzeppelin-v4 
      que solicita flashloans usando el protocolo AAVE y ejecuta trades(o swaps) entre uniswap-v3 y quickswap-v3

    \item Un \href{https://github.com/TheSnakeWitcher/mypeople}{addressbook} para manejar contactos escrito en rust usando clap para la CLI y sqlite como base de datos

    \item \href{https://github.com/TheSnakeWitcher/hardhat-inspect.git}{Plugin para el framework hardhat} usando typescript para de forma conveniente acceder
      a ciertos datos de un proyecto hardhat y hacer el proceso de desarrollo mas facil y fluido
\resumeItemListEnd


\section{\textbf{Educacion}}
\vspace{-0.4mm}
\resumeSubHeadingListStart
\resumeSubheading
{Universidad Tecnologica de La Habana Jose Antonio Echeverria}{Habana, Cuba}
{Ingenieria Automatica}{2017 - 2022}
\resumeSubHeadingListEnd
\vspace{-6mm}

\end{document}
