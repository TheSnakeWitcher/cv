\documentclass[a4paper,11pt]{article}
\usepackage{latexsym}
\usepackage{xcolor}
\usepackage{float}
\usepackage{ragged2e}
\usepackage[empty]{fullpage}
\usepackage{wrapfig}
\usepackage{lipsum}
\usepackage{tabularx}
\usepackage{titlesec}
\usepackage{geometry}
\usepackage{marvosym}
\usepackage{verbatim}
\usepackage{enumitem}
\usepackage{fancyhdr}
\usepackage{multicol}
\usepackage{graphicx}
\usepackage{cfr-lm}
\usepackage[T1]{fontenc}
\usepackage{fontawesome5}

\usepackage[hidelinks]{hyperref}
\hypersetup{
    colorlinks=true,
    linkcolor=darkblue,
    filecolor=darkblue,
    urlcolor=darkblue,
}

\usepackage[most]{tcolorbox}
\tcbset{
    frame code={},
    center title,
    left=0pt,
    right=0pt,
    top=0pt,
    bottom=0pt,
    colback=gray!20,
    colframe=white,
    width=\dimexpr\textwidth\relax,
    enlarge left by=-2mm,
    boxsep=4pt,
    arc=0pt,outer arc=0pt,
}


\definecolor{darkblue}{RGB}{0,0,139}
\setlength{\multicolsep}{0pt} 
\pagestyle{fancy}
\fancyhf{}
\fancyfoot{}
\renewcommand{\headrulewidth}{0pt}
\renewcommand{\footrulewidth}{0pt}
\geometry{left=1.4cm, top=0.8cm, right=1.2cm, bottom=1cm}
\setlength{\footskip}{5pt}
\urlstyle{same}
\raggedright
\setlength{\tabcolsep}{0in}
\titleformat{\section}{
  \vspace{-4pt}\scshape\raggedright\large
}{}{0em}{}[\color{black}\titlerule \vspace{-7pt}]

\newcommand{\resumeItem}[2]{
    \item { \textbf{#1}{\hspace{0.5mm}#2 \vspace{-0.5mm}} }
}

\newcommand{\resumePOR}[3]{
\vspace{0.5mm}\item
    \begin{tabular*}{0.97\textwidth}[t]{l@{\extracolsep{\fill}}r}
        \textbf{#1}\hspace{0.3mm}#2 & \textit{\small{#3}} 
    \end{tabular*}
    \vspace{-2mm}
}

\newcommand{\resumeSubheading}[4]{
\vspace{0.5mm}\item
    \begin{tabular*}{0.98\textwidth}[t]{l@{\extracolsep{\fill}}r}
        \textbf{#1} & \textit{\footnotesize{#4}} \\
        \textit{\footnotesize{#3}} &  \footnotesize{#2}\\
    \end{tabular*}
    \vspace{-2.4mm}
}

\newcommand{\resumeProject}[4]{
\vspace{0.5mm}\item
    \begin{tabular*}{0.98\textwidth}[t]{l@{\extracolsep{\fill}}r}
        \textbf{#1} & \textit{\footnotesize{#3}} \\
        \footnotesize{\textit{#2}} & \footnotesize{#4}
    \end{tabular*}
    \vspace{-2.4mm}
}

\newcommand{\resumeSubItem}[2]{\resumeItem{#1}{#2}\vspace{-4pt}}

\renewcommand{\labelitemi}{$\vcenter{\hbox{\tiny$\bullet$}}$}
\renewcommand{\labelitemii}{$\vcenter{\hbox{\tiny$\circ$}}$}

\newcommand{\resumeSubHeadingListStart}{\begin{itemize}[leftmargin=*,labelsep=1mm]}
\newcommand{\resumeHeadingSkillStart}{\begin{itemize}[leftmargin=*,itemsep=1.7mm, rightmargin=2ex]}
\newcommand{\resumeItemListStart}{\begin{itemize}[leftmargin=*,labelsep=1mm,itemsep=0.5mm]}

\newcommand{\resumeSubHeadingListEnd}{\end{itemize}\vspace{2mm}}
\newcommand{\resumeHeadingSkillEnd}{\end{itemize}\vspace{-2mm}}
\newcommand{\resumeItemListEnd}{\end{itemize}\vspace{-2mm}}
    \newcommand{\cvsection}[1]{
    \vspace{2mm}
    \begin{tcolorbox}
        \textbf{\large #1}
    \end{tcolorbox}
    \vspace{-4mm}
}

\newcolumntype{L}{>{\raggedright\arraybackslash}X}
\newcolumntype{R}{>{\raggedleft\arraybackslash}X}
\newcolumntype{C}{>{\centering\arraybackslash}X}

\newcommand{\socialicon}[1]{\raisebox{-0.05em}{\resizebox{!}{1em}{#1}}}
\newcommand{\ieeeicon}[1]{\raisebox{-0.3em}{\resizebox{!}{1.3em}{#1}}}

\newcommand{\headerfonti}{\fontfamily{phv}\selectfont}
\newcommand{\headerfontii}{\fontfamily{ptm}\selectfont}
\newcommand{\headerfontiii}{\fontfamily{ppl}\selectfont}
\newcommand{\headerfontiv}{\fontfamily{pbk}\selectfont}
\newcommand{\headerfontv}{\fontfamily{pag}\selectfont}
\newcommand{\headerfontvi}{\fontfamily{cmss}\selectfont}
\newcommand{\headerfontvii}{\fontfamily{qhv}\selectfont}
\newcommand{\headerfontviii}{\fontfamily{qpl}\selectfont}
\newcommand{\headerfontix}{\fontfamily{qtm}\selectfont}
\newcommand{\headerfontx}{\fontfamily{bch}\selectfont}


\begin{document}
\headerfontiii

\begin{center} {\Huge\textbf{Alejandro Virelles}} \end{center}
\vspace{-6mm}

\begin{center}
    \small{
        \socialicon{ \href{mailto:thesnakewitcher@gmail.com}{\color{black}{\faEnvelope}} }
        \socialicon{ \href{https://github.com/TheSnakeWitcher}{\color{black}{\faGithub}} }
        \socialicon{ \href{https://t.me/TheSnakeWitcher}{\color{black}{\faTelegram}} } {}
    }
\end{center}
\vspace{-6mm}

\vspace{-4mm}
\section{\textbf{Habilidades}}\vspace{-0.4mm}
\resumeHeadingSkillStart
    \resumeSubItem{Lenguajes de Programacion:}
        { plc/ladder, c/c++, rust, go, typescript, python, sql, bash, vhdl }
    \resumeSubItem{Bases de datos:}
        { postgres, sqlite, mongodb, redis }
    \resumeSubItem{Herramientas \& Tecnologias:}
        { linux, matlab, excel, git, github, docker, aws }
    \resumeSubItem{Lenguajes:}
        { español, ingles }
\resumeHeadingSkillEnd


\section{\textbf{Experiencia}}
\vspace{-0.4mm}
\resumeSubHeadingListStart
\resumeSubheading
    { {\href{https://portal.fusy.app}{\color{black}{Fusyona}} }}{}
    {Mid Senior Blockchain developer}{Junio 2024 - Abril 2025}

    \resumeItemListStart
        \item Desarrollo de contractos inteligentes con solidity siguiendo las metodologias TDD y SCRUM para
          testing con hardhat framework usando mocha, chai, hardhat-chai-matchers, etc 

        \item Mentoria de juniors para que se vuelvan rapidamente productivos, conscienctes de la seguridad, que sigan buenas practicas
          y desarrollen confianza para contribuir al projecto

        \item Desarrollo de una collection ERC721 crosschain como early access key para el juego fusyfox
          usando chainlink CCIP como protocolo de interoperabilidad and creando un token ERC20 crosschain
          con soporte para payable token(ERC1363) que usa LayerZero-v1 como protocolo de interoperabilidad

        \item Construyendo una \href{https://github.com/Fusyona/web3-core}{libreria} en typescript para crear facilmente SDKs para interactuar facilmente
          con los contratos inteligentes desde el front-end y adicionalmente contiene logica para asginar ID
          a colecciones que lo necesitan(como ERC721 y ERC1155)

        \item Refactorizando un launchpad para hacer el contracto desplegable(para cumplir el limite de tamaño de contrato de spurious dragon),
          mas mantenible y eficiente con el gas al usar minimal proxy(EIP 1167), custom errors y reducir la informacion
          almacenada on-chain en el contracto al tomar ventaja de los eventos

        \item Experiencia con patrones de solidity para securidad(check-effects-interaction, pull over push,
          access restriction, emergency stop or pausable contracts), proxies(transparent, UUPS, beacon)
          y upgradeabilidad(diamond)

        \item Estableciendo un flujo basico de CI usando github actions que detectas vulnerabilidades al crear
          PRs usando slither y que ejecuta la test suite al hacer merge
    \resumeItemListEnd 
\resumeSubHeadingListEnd
\vspace{-6mm}
\vspace{-4mm}
\resumeSubHeadingListStart
\resumeSubheading
    { {\color{black}{ATI} }}{}
    {Ingeniero en Automatica}{Diciembre 2023 - Mayo 2024}

    \resumeItemListStart
        \item Inspeccion simple y basica de parques fotovoltaicos
        \item Construyendo un servidor en golang para comunicar la base de datos de un servicio de monitoreo con software legacy
    \resumeItemListEnd 
\resumeSubHeadingListEnd
\vspace{-6mm}


\section{\textbf{Proyectos}}
\vspace{0.4mm}
\resumeItemListStart
    \vspace{0.8mm}
    \item Programando PLCs Modim de Schneider Electric usando ladder en EcoStruxture

    \item Implementando sistemas digitales avanzados con VHDL en FPGA xilinx spartan 

    \item Diseño de PCB con KiCad y Proteus

    \item Desarrollo de firmware para placas como arduino y raspberry pi

    \item Programando una \href{https://vesting.cooltech.quest}{platforma decentralizada de vesting} para tokens erc20 con distribucion automatica que integra
      blockchain y la creacion dinamica de recursos en AWS

    \item Construyendo un \href{https://github.com/TheSnakeWitcher/AuctionHub-server}{servidor simple} en golang que usa el framework web echo para hacer
      operaciones CRUD a traves de API REST al interactuar con un servicio de subasta

    \item Creando un \href{https://github.com/TheSnakeWitcher/mypeople-go}{addressbook} en go con CLI(usando cobra), configuracion(usando viper)
      y sqlite como base de datos(usando sqlc)

    \item CLI basica para un \href{https://github.com/TheSnakeWitcher/mypeople}{addressbook} para manejar contactos escirta en rust usando clap y sqlite
\resumeItemListEnd


\section{\textbf{Educacion}}
\vspace{-0.4mm}
\resumeSubHeadingListStart
\resumeSubheading
{Universidad Tecnologica de la Habana Jose Antonio Echeverria}{Habana, Cuba}
{Ingenieria en Automatica}{2017 - 2022}
\resumeSubHeadingListEnd
\vspace{-6mm}

\end{document}
